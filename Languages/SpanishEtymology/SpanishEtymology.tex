\documentclass[10pt]{book}
\usepackage[utf8]{inputenc}
\usepackage{fontenc}
\usepackage[spanish,english]{babel}
% Greek text support - using Unicode directly with fontspec
\usepackage{amsmath,amssymb}
\usepackage{enumitem}
\usepackage{titlesec}
\usepackage{tcolorbox}
\usepackage{needspace}

\usepackage{geometry}



%\setmainfont{Latin Modern Roman}

\titleformat{\section}{\normalfont\LARGE\sffamily\bfseries}{\hspace{0em}}{0em}{\nopagebreak[4]}
\titleformat{\subsection}{\normalfont\sffamily\bfseries}{\hspace{0em}}{0em}{\nopagebreak[4]}
%\titleformat{\subsection}{\normalfont\sffamily\bfseries}{\hspace{0em}}{0em}{}

\setcounter{tocdepth}{1}

% Reduce line spacing
\linespread{1.1}

% Reduce paragraph spacing
\setlength{\parskip}{0.2em}

% Reduce itemize spacing
\setlength{\itemsep}{0.1em}
\setlength{\parsep}{0.1em}

% Reduce section spacing
\titlespacing{\section}{0pt}{1.5em}{0.2em}
\titlespacing{\subsection}{0pt}{0.3em}{0.1em}

\setlength{\parindent}{0pt}

% Prevent orphaned sections and subsections
\widowpenalty=10000
\clubpenalty=10000
\displaywidowpenalty=10000
\interlinepenalty=10000
\brokenpenalty=10000

% Define etymology block
\newtcolorbox{etymologybox}[1][]{
	colback=white,
	% colback=gray!5,
	colframe=black!25,
	fonttitle=\bfseries,
	title=Etymology,
	boxsep=0.3em,
	left=0.5em,
	right=0.5em,
	top=0.3em,
	bottom=0.3em,
	#1
}

\newcommand{\wordentry}[2]{
	\large #1
	\vspace{-0.5em}
	\begin{etymologybox}
		#2
	\end{etymologybox}
}

% Command to ensure enough space before a section
\newcommand{\ensurespace}[1][6]{
	\needspace{#1\baselineskip}
}

% Redefine section to include space checking
\let\oldsection\section
\renewcommand{\section}[1]{
	\needspace{8\baselineskip}
	\oldsection{#1}
}


\title{\Huge Etymología del Español\\\large Análisis y Origen de Palabras Seleccionadas}
\author{Wang Shawn}
\date{\today}

\begin{document}
	\maketitle
	
	\clearpage
	\tableofcontents
	
	\newpage
	
	\newgeometry{
		inner=3.0cm,
		outer=2.5cm,
		top=2.5cm,
		bottom=2.0cm,
		bindingoffset=1.0cm
	}
	
	\chapter{Palabras Comunes}
	
	\section{tarjeta}
	\wordentry{card, ticket}{
	Proviene del francés \textit{targette} (cerrojo pequeño), diminutivo de \textit{targe} (escudo). El significado moderno de "tarjeta" como pedazo de cartón o plástico proviene de su uso como identificador.
	\begin{itemize}
		\item From Old French: targette
		
		This is a diminutive of targe, which meant a small shield. The suffix -ette is a French diminutive, so targette literally meant little shield.
		
		\item From Latin: targa (Medieval Latin)
		
		This term referred to a shield and is the ancestor of both French targe and Italian targa (which today can mean "license plate").
	\end{itemize}
	}
	
	\subsection*{Evolution}
	
	\begin{itemize}
		\item Originally, the word referred to a \textbf{small shield}.
		\item Over time, the sense evolved metaphorically to mean a \textbf{small, flat object}, which is the shape of many cards.
		\item By the time it entered Spanish (as tarjeta), it came to refer to things like:
		\begin{itemize}
			\item a \textbf{visiting card}
			\item a \textbf{business card}
			\item a \textbf{postcard}
			\item later, \textbf{credit cards}, \textbf{ID cards}, etc.
		\end{itemize}
	\end{itemize}
	
	\subsection*{Summary}
	
	\textbf{tarjeta} = targette (Old French, "little shield") $\rightarrow$ from \textbf{targe} (shield) $\rightarrow$ from \textbf{targa} (Medieval Latin).
	Over time, the term shifted from "small shield" to "small flat card."
	
	
	\section{cuchillo}
	
	\wordentry{knife}{
	Deriva del latín vulgar *\textit{cultellus}, diminutivo de \textit{culter} (cuchillo grande, cuchilla para sacrificar animales).
	\begin{itemize}
		\item From Late Latin: cultellus
		\begin{itemize}
			\item A diminutive of culter, meaning knife, blade, plowshare
			\item So cultellus = little knife
		\end{itemize}
	\end{itemize}
	}
	
	\subsection*{Latin Components}
	
	\begin{itemize}
		\item culter = knife, blade (also used for plow blades or sacrificial knives)
		
		\item ellus = Latin diminutive suffix
		
		\item $\rightarrow$ cultellus = small knife, which over time evolved phonetically into cuchillo in Old Spanish
	\end{itemize}
	
	\subsection*{Related words in other languages}
	\begin{itemize}
		\item Italian: coltello
		\item French: couteau
		\item Portuguese: cutelo (a cleaver), faca (general term for knife)
	\end{itemize}
	
	\subsection{Summary}
	cuchillo $\leftarrow$ cultellus (Late Latin, "small knife") $\leftarrow$ culter (Latin, "knife, blade")
	
	$\Rightarrow$ Over time, the word shifted phonologically and semantically to the familiar Spanish term for knife.
	
	\section{efectivo}
	\wordentry{cash, effective}{
	Proviene del latín \textit{effectivus} ("que produce un efecto"), del verbo \textit{efficere} ("llevar a cabo").
	
	\begin{itemize}
		\item From Latin: effectīvus
		\begin{itemize}
			\item Meaning: productive, effective, that causes something to happen
			\item Rooted in the verb efficere = to bring about, to accomplish, to produce
		\end{itemize}
	\end{itemize}
	}
	
	\subsection{Latin Components}
	
	\begin{itemize}
		\item ex- = out, from
		\item facere = to do, to make
		\item $\rightarrow$ efficere = to bring into being, to make happen
		\item $\rightarrow$ effectus = result, effect, outcome
		\item $\rightarrow$ effectīvus = producing an effect
	\end{itemize}
	
	\subsection{Evolution}
	
	\begin{itemize}
		\item Originally, efectivo referred to something real, actual, as opposed to theoretical.
		\item Then it came to mean effective — something that works or has an effect.
		\item Later, in military and administrative contexts, efectivo referred to actual available personnel or resources (i.e., real, present, not just on paper).
		\item Finally, in finance, it came to refer to cash (as opposed to credit), i.e., real money in hand.
	\end{itemize}
	
	\subsection{Summary}
	efectivo $\leftarrow$ effectīvus (Latin, "productive, real, that causes something") $\leftarrow$ efficere ("to accomplish") $\leftarrow$ ex- + facere ("to make").
	
	$\Rightarrow$ Over time, the word shifted from "producing a result" $\rightarrow$ "real/present" $\rightarrow$ "cash on hand".

\chapter{Color}

	\section{title}
\chapter{Relacionadas con Comida}
	
	\section{Cuchara}
	\wordentry{spoon}{
	Del latín tardío \textit{cochlear}, que designaba una cuchara o concha. Se relaciona con \textit{cochlea} (caracol).
	\begin{itemize}
		\item From Late Latin: cochleārium
		
		This was a \textbf{type of spoon} used in Roman times, especially for eating snails (Latin cochlea = snail or spiral).
		
		\item From Latin: cochlea
		This originally meant \textbf{snail} or \textbf{spiral}, referring to the shape of a snail shell.
	\end{itemize}
	}
	
	\subsection*{Evolution}
	\begin{itemize}
		\item cochlea (snail/spiral) → cochlearium (spoon for eating snails)
		
		\item In Vulgar Latin, cochlearium gradually evolved phonetically and morphologically into \textbf{cuchara} in Spanish.
	\end{itemize}
	
	\subsection*{Related words in other languages}
	
	\begin{itemize}
		\item Portuguese: colher
		
		\item Italian: cucchiaio
		
		\item French: cuillère
	\end{itemize}
	
	\subsection*{Summary}
	
	cuchara $\leftarrow$ cochleārium (Late Latin, "spoon for snails") $\leftarrow$ cochlea (Latin, "snail, spiral")
	
	$\Rightarrow$ The name likely came from the curved, spiral-like shape of early spoons resembling snail shells.
	
	\section{Bife}
	\wordentry{steak}{
		Proviene del inglés \textit{beefsteak}, que a su vez deriva de las palabras \textit{beef} (carne de res) y \textit{steak} (trozo de carne). El término fue adoptado por el español, especialmente en América Latina.
		
		\begin{itemize}
			\item From English: beefsteak
			
			\textit{Beefsteak} comes from the combination of \textbf{beef} (meat of a cow) and \textbf{steak} (a piece of meat, particularly one that is cut and usually grilled or fried).
			
			\item From Old French: \textit{bœuf} (beef)
			
			Old French \textit{bœuf} (from Latin bovem) means \textbf{ox, bull}, and gave rise to the modern English word beef.
		\end{itemize}
	}
	
	\subsection*{Evolution}
	\begin{itemize}
		\item \textit{beef} (meat of a cow) + \textit{steak} (a cut of meat) → \textit{beefsteak} (a specific term for a piece of beef)
		\item The term was adopted into \textbf{Spanish} during the \textbf{colonial era}, particularly in regions influenced by English-speaking cultures, like the Caribbean.
		\item \textit{Bife} evolved as a local term used widely across Latin America, especially in countries like Argentina and Uruguay, to refer to \textbf{steak} or \textbf{beef cuts}.
	\end{itemize}
	
	\subsection*{Related words in other languages}
	\begin{itemize}
		\item Portuguese: bife (direct borrowing from Spanish)
		\item English: beefsteak
		\item French: bifteck (borrowed from English \textit{beefsteak})
	\end{itemize}
	
	\subsection*{Summary}
	\textbf{bife} $\leftarrow$ \textit{beefsteak} (English, "beef + steak") $\leftarrow$ \textit{bœuf} (Old French, "beef")
	
	$\Rightarrow$ Adopted from English, \textit{bife} refers to a \textbf{steak} or \textbf{cut of beef}, especially popular in Latin American cuisine.
	

	\section{Zanahoria}
	\wordentry{carrot}{
		Proviene del árabe hispánico \textit{safunnāriyya}, que a su vez proviene del árabe clásico \textit{ṣafannāriyya}, que designaba la zanahoria. Esta palabra está relacionada con el griego \textit{σαφηνάριον} (saphēnárion).
		\begin{itemize}
			\item From Arabic: safunnāriyya
			
			The Arabic word \textit{safunnāriyya} referred to the \textbf{carrot}, and was used in Al-Andalus (Muslim Spain) to describe this vegetable.
			
			\item From Greek: \textit{σαφηνάριον} (saphēnárion)
			
			This Greek word, which referred to a \textbf{type of root vegetable}, is the origin of the Arabic term.
			
			\item From Ancient Greek: \textit{saphēs} (clear, evident), possibly referring to the bright orange color of the carrot.
		\end{itemize}
	}
	
	\subsection*{Evolution}
	\begin{itemize}
		\item The word \textit{saphēnárion} (Greek) evolved into \textit{safunnāriyya} in Arabic.
		\item The term \textit{safunnāriyya} was borrowed into Old Spanish during the Muslim rule of the Iberian Peninsula.
		\item In modern Spanish, it evolved into \textbf{zanahoria}, which refers specifically to the orange root vegetable.
	\end{itemize}
	
	\subsection*{Related words in other languages}
	\begin{itemize}
		\item Portuguese: cenoura (from Latin \textit{caenorūca})
		\item French: carotte (from Latin \textit{carota})
		\item Italian: carota (from Latin \textit{carota})
		\item English: carrot (from Old French \textit{carotte})
	\end{itemize}
	
	\subsection*{Summary}
	\textbf{zanahoria} $\leftarrow$ \textit{safunnāriyya} (Arabic, "carrot") $\leftarrow$ \textit{σαφηνάριον} (Greek, "root vegetable")
	
	$\Rightarrow$ The term evolved from Arabic to Spanish and came to refer specifically to the \textbf{carrot}, which was introduced into Spain through contact with the Arab world.

	\section{Frijol}
	\wordentry{bean}{
	Del náhuatl \textit{etl} (frijol), a través del español caribeño y del contacto con lenguas indígenas mesoamericanas.
	\begin{itemize}
		\item Modern Spanish: frijol (plural: frijoles)
		
		\item From Old Spanish: fesol
		
		\item From Latin: phaseolus
		
		\begin{itemize}
			\item Diminutive of Greek phasēlos (φάσηλος), meaning bean or kidney bean
			
			\item The Greek term itself may have referred to a specific type of legume, possibly the cowpea or black-eyed pea
		\end{itemize}
	\end{itemize}
	}
	
	\subsection{Sound Evolution}
	
	\begin{itemize}
		\item Latin phaseolus $\rightarrow$ Old Spanish fesol
		\item In Spain, fesol evolved into judía (modern word for bean in Spain)
		\item In Mexico and Central America, the word evolved into frijol, with an influence likely from Caribbean or Mozarabic dialects
	\end{itemize}
	
	\subsection{Regional Variation}
	
	\begin{itemize}
		\item Latin America: frijol is the standard word for bean (especially kidney beans, black beans, pinto beans)
		\item Spain: judía is more commonly used now
		\item Catalan: fesol still survives as the word for bean
		\item Portuguese: feijão
		\item Italian: fagiolo
	\end{itemize}

	All of these come from the same Latin root: phaseolus.
	
	
	\subsection{Cultural Importance}
	
	\begin{itemize}
		\item In Mexican, Central American, and Caribbean cuisines, frijoles are essential:
		\begin{itemize}
			\item Frijoles negros (black beans)
			\item Frijoles pintos (pinto beans)
			\item Frijoles refritos (refried beans)
		\end{itemize}
		\item The dish "arroz con frijoles" (rice and beans) is a staple in many countries	
		\item Frijoles are not just food — they're a symbol of sustenance and identity in many cultures
	\end{itemize}

	\subsection{Summary}
	Spanish (LA), Portuguese, Italian, and Catalan derive their word for 'bean' from the Latin word phaseolus. However, Spanish (Spain) is an exception, using the word judía, which is a later adoption and does not stem from phaseolus.
	
	\section{Cerveza}
	\wordentry{beer}{
	Del latín \textit{cerevisia}, probablemente de origen celta, relacionado con \textit{Ceres} (diosa del grano).
	\begin{itemize}
		\item Spanish: cerveza
		\item From Medieval Latin: cerevisia
		\item Possibly from Gaulish (Celtic): cervesia or cervisia
		
		Related to Ceres, the Roman goddess of agriculture and grain
	\end{itemize}

	}

	\subsection{Evolution}
	
	\begin{itemize}
		\item Latin: cerevisia
		
		\begin{itemize}
			\item Used by Roman authors to refer to fermented grain drinks (beer), especially in Celtic and Germanic provinces
			\item In contrast, Romans preferred wine and regarded beer as more "barbaric"
		\end{itemize}
		\item The name is likely derived from the name of the goddess Ceres (grain) + a suffix -visia (possibly related to fermentation or drink)
	\end{itemize}
	
	\subsection*{Words in other languages}
	
	\begin{itemize}
		\item Portuguese: cerveja (Latin \textit{cerevisia})
		
		\item German: Bier (Proto-Germanic \textit{beuzą})
		
		\item French: bière (From Germanic bier)
	\end{itemize}
	
	\subsection{Summary}
	
	cerveza $\leftarrow$ Latin cerevisia $\leftarrow$ possibly Celtic root, related to Ceres (grain goddess)
	$\Rightarrow$ Indicates an ancient grain-based fermented drink
	$\Rightarrow$ The word survived in Iberian Romance languages but was replaced in others by Germanic influence
	
	\section{saludable}
	\wordentry{healthy}{
		Proviene del latín \textit{salūdābilis}, que significa “saludable” o “beneficioso para la salud”, derivado de \textit{salūs} (salud) y el sufijo \textit{-ābilis}, que indica capacidad o posibilidad.
		\begin{itemize}
			\item From Latin: \textit{salūdābilis}
			
			Formed from \textit{salūs} (health, well-being) + \textit{-ābilis} (able to, suitable for), meaning \textbf{conducive to health} or \textbf{promoting well-being}.
			
			\item From Latin: \textit{salūs}
			
			Meaning \textbf{health}, \textbf{welfare}, or \textbf{safety}. It is also the root of words like \textit{salvation} and \textit{salute} in English.
		\end{itemize}
	}
	
	\subsection*{Evolution}
	\begin{itemize}
		\item \textit{salūs} (health) + \textit{-ābilis} (able to) → \textit{salūdābilis} (that promotes health)
		\item In Old Spanish, this became \textit{saludable}, preserving both form and meaning.
		\item In modern Spanish, \textit{saludable} means \textbf{healthy} (especially in reference to habits, food, and lifestyles).
	\end{itemize}
	
	\subsection*{Related words in other languages}
	\begin{itemize}
		\item Portuguese: saudável
		\item Italian: salutare
		\item French: salubre (more formal), \textit{sain} (common)
		\item English: salubrious (formal), \textit{healthy}
	\end{itemize}
	
	\subsection*{Summary}
	\textbf{saludable} $\leftarrow$ \textit{salūdābilis} (Latin, "conducive to health") $\leftarrow$ \textit{salūs} (Latin, "health, well-being")
	
	$\Rightarrow$ The word has retained its core meaning over centuries: something that promotes or preserves \textbf{health and well-being}.
	
	\chapter{Profesión}
	
	\section{Camarero}
	\wordentry{waiter, bartender}{
		Proviene del latín \textit{cammerarius}, que significaba "encargado de la cámara" o "sirviente de la cámara". La palabra está relacionada con \textit{camerarius}, que a su vez proviene de \textit{camera} (habitación o cámara).
		\begin{itemize}
			\item From Late Latin: cammerarius
			
			This referred to a chamberlain or steward in charge of a chamber or room. The term evolved in medieval times to refer to someone in charge of the service in a royal or noble household.
			
			\item From Latin: camera
			
			Originally meaning chamber, this term referred to a room or a space, which is the source of both \textit{camarero} and the modern English word \textit{chamber}.
		\end{itemize}
	}
	
	\subsection*{Evolution}
	\begin{itemize}
		\item camera (chamber) → \textbf{camerarius} (steward, chamberlain)
		\item camerarius evolved into \textbf{cammerarius} in Vulgar Latin.
		\item By the time it entered Spanish as \textbf{camarero}, it came to refer to a \textbf{servant} or \textbf{waiter} who serves in a chamber or dining area.
	\end{itemize}
	
	\subsection*{Related words in other languages}
	\begin{itemize}
		\item Portuguese: camareiro (waiter, chamberlain)
		\item Italian: cameriere (waiter)
		\item French: camerier (historical term for a servant, but not commonly used today)
	\end{itemize}
	
	\subsection*{Summary}
	\textbf{camarero} $\leftarrow$ cammerarius (Late Latin, "chamberlain, steward") $\leftarrow$ camera (Latin, "chamber, room")
	
	$\Rightarrow$ Originally referring to a servant in charge of a chamber or room, the term evolved to mean \textbf{waiter} or \textbf{bartender} in modern Spanish.
	
	%\restoregeometry  % Restore original geometry if needed
\end{document}

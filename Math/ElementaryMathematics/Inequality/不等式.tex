\documentclass[a4paper,10pt]{ctexbook}
\usepackage{amsmath,amssymb, amsfonts}
\usepackage{exscale}
\usepackage{xeCJK}
\usepackage{indentfirst}

\title{不等式}

\begin{document}

\maketitle
\tableofcontents

\newpage

\chapter{均值不等式}

\section{$\sqrt{ab} \leqslant \frac{a+b}{2}$}

若$ a > 0 $,$ b > 0 $,则$ \sqrt{ab} \le \frac{a+b}{2} $

\subsection{证明}

\[ (c-d)^2 \ge 0 \]
\[ c^2 + d^2 - 2cd \ge 0 \]
\[ c^2 + d^2 \ge 2cd \]
\[ \frac{c^2 + d^2}{2} \ge cd \]
令$ c^2 = a $,$ d^2 = b $,得$ c = \sqrt{a}$,$ d = \sqrt{b} $.

于是有
\[ \frac{a+b}{2} \ge \sqrt{a} \cdot \sqrt{b} = \sqrt{ab} \]

\subsection{使用规则}

1正2定3相等。

“正”的意思是$a$和$b$都是正数。

“定”的意思是加和($a + b$)和乘积($a \cdot b$)是定值,即具体的值,有了具体的值就能确定范围。如果不是定值,例如$ 2 \sqrt{\frac{1}{x}} $,那么意义就不大了。

“相等”的意思是当$a$和$b$相等时(例如$x$=$\frac{1}{x}$时),均值不等式能取到等号。这一点可以从$ (c-d)^2 \ge 0 $看出。

\section{均值不等式的高级形态}

\[ \sqrt{ab} \le \frac{a+b}{2} \le \sqrt{ \frac{a^2 + b^2}{2} }\]

此形态在高中数学中有个重要的作用——三态转化。“三态”为“乘积”、“加和”以及“平方和”。有了该高级形态后,加和、乘积、平方和之间可以随意转换。

\section{例题}

\subsection{“1”的替换}

例题1:设$a>0$,$b>0$,若$ a+b=1 $,则$ \frac{1}{a} + \frac{1}{b} $的最小值是(    )

A.2 \quad B.$\frac{1}{4}$ \quad  C.4 \quad D. 8

解析:因为$\frac{1}{a}+\frac{1}{b}=(\frac{1}{a}+\frac{1}{b})\cdot1$,$a+b=1$,
所以$\frac{1}{a}+\frac{1}{b}=(\frac{1}{a}+\frac{1}{b})\cdot(a+b)=1+\frac{b}{a}+\frac{a}{b}+1=2+\frac{b}{a}+\frac{a}{b}$

因为$\frac{b}{a}+\frac{a}{b} \ge 2\sqrt{\frac{b}{a}\cdot\frac{a}{b}}=2$,所以$\frac{b}{a}+\frac{a}{b} \ge 2$,

所以$\frac{1}{a}+\frac{1}{b}$的最小值为4. 选C.

% 参考https://www.bilibili.com/video/BV1tL4y1W7w5


\end{document}

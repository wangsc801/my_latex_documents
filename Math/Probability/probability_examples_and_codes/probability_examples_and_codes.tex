\documentclass[a4paper]{ctexbook}

% Packages
\usepackage{inputenc}
\usepackage{amsmath}  % For math formulas
\usepackage{amssymb}  % For additional math symbols
\usepackage{amsfonts} % For more math fonts

\usepackage{geometry} % For adjusting page margins
\geometry{left=3.18cm,right=3.18cm,top=2.54cm,bottom=2.54cm}

\usepackage{fontspec}
\usepackage{multicol}

%\setmainfont[
%    Path = ./fonts/, % Assuming your font is in a 'fonts' subfolder
%    Extension = .otf,
%    UprightFont = *-Regular,
%    BoldFont = *-Bold,
%{SourceHanSerifSC}

\setCJKmainfont[
    Path = ./fonts/, % Assuming your font is in a 'fonts' subfolder
    Extension = .ttf,
    UprightFont = *-Regular,
    BoldFont = *-Semibold,
]{MiSans}

\setmainfont[
    Path = ./fonts/, % Assuming your font is in a 'fonts' subfolder
    Extension = .ttf,
    UprightFont = *-Regular,
    BoldFont = *-Semibold,
]{MiSans}


\setmonofont[
    Path=./fonts/, % Adjust this path if your fonts folder is elsewhere
    Extension=.ttf, % Specify the file extension
    UprightFont=*-Regular, % Use FiraCode-Regular.ttf for normal weight
    BoldFont=*-Bold,       % Use FiraCode-Bold.ttf for bold weight (if you have it)
    Contextuals=Alternate % This is crucial for Fira Code ligatures
]{FiraCode}


\usepackage{hyperref} % For hyperlinks within the document

\usepackage{enumitem} % Provides enhanced control over lists

\usepackage{titlesec} % Customizes section headings
% Customize the format of section headings. Set the font to normal, large size, and bold.
\titleformat{\section}{\normalfont\Large\bfseries}{}{0em}{}

% \usepackage[BoldFont,SlantFont]{xeCJK} % Mainly used to support Chinese characters
%\usepackage{xeCJK}

\usepackage{mdframed} % Creates framed boxes around text

\usepackage{minted}

\usepackage{setspace}
\setstretch{1.25}

\newcommand{\boldtitle}[1]{%
    \par % Ensure it starts on a new line (new paragraph)
    \setstretch{1.75}% Apply line spacing for this paragraph
    ▫︎ \textbf{#1}\par% Add the symbol, bold text, and end the paragraph
    \setstretch{1.0}% Reset line spacing for subsequent text
}

\newcommand{\separatorline}{
	\vspace{1em}
	\centerline{\rule{0.6\linewidth}{0.5pt}}
	\vspace{1em}
}


\begin{document}

\chapter{组合}

\section{经典例题}

\subsection*{不放回}

一个盒子中有4个红球和6个白球,现从中不放回地随机抽取3个球,求恰好抽到2个红球和1个白球的概率。

解答过程:

\begin{enumerate}
\item 
计算总事件数:
\[
C(10,3) = \frac{10!}{3!(10-3)!} = \frac{10 \times 9 \times 8}{3 \times 2 \times 1} = 120
\]

\item 
计算有利事件数:
\[
C(4,2) \times C(6,1) = \frac{4!}{2!2!} \times \frac{6!}{1!5!} = 6 \times 6 = 36
\]

\item 
计算概率:
\[
P(A) = \frac{36}{120} = \frac{3}{10}
\]
\end{enumerate}

答案:恰好抽到2个红球和1个白球的概率为 \(\boxed{\dfrac{3}{10}}\)。

\subsubsection*{代码模拟}

有红、黄、蓝、绿、紫5种颜色的球各一个,有放有回地拿球3次,求至少抽到一个红球的概率。

\begin{minted}{python}
import random

def simulate_draws(num_simulations):
    colors = ['红', '黄', '蓝', '绿', '紫']  # 球的颜色
    success_count = 0  # 记录成功次数(至少抽到一次红球)

    for _ in range(num_simulations):
        draws = [random.choice(colors) for _ in range(3)]  # 抽取3次
        if '红' in draws:  # 如果至少有一次是红色
            success_count += 1

    probability = success_count / num_simulations  # 成功的概率
    return probability

# 自定义模拟次数
num_simulations = int(input("请输入模拟次数:"))
p = simulate_draws(num_simulations)

print(f"在{num_simulations}次模拟中,至少抽到一次红球的概率为:{p:.4f}")
\end{minted}

\subsubsection*{一些疑问}
对于计算组合数$C_k^n$, 我觉得用这种算法比较简单和快速: $\frac{n(n-1) \cdots (n-k+1)}{k(k-1) \cdots 1}$.
但是为什么很多教科书里以这个公式为主:$C_k^n=\frac{n!}{k!(n-k)!}$.

你提到的算法和教科书中的公式是\textbf{数学等价的}。它们只是表达方式不同,但计算结果一样。之所以教科书更常用后一种形式,是因为教科书强调公式的\textbf{一般性和简洁性},更希望用\textbf{统一的符号表达复杂问题}。例如:

\begin{itemize}
\item 教科书里经常推导组合数的性质(如对称性 $C_k^n = C_{n-k}^n$\: 、递推公式 $C_k^n = C_{k-1}^{n-1} + C_k^{n-1}$),使用阶乘形式更容易进行代数操作。
\item 用阶乘定义便于形式化定义,比如:
\end{itemize}

  $$
  C_k^n = \frac{n!}{k!(n-k)!}, \quad \text{其中 } 0 \le k \le n
  $$

\chapter{概率论基础}

\section{事件}

设$A$,$B$,$C$表示三个随机事件,试将下列事件用$A$,$B$,$C$表示出来.

\begin{multicols}{2}
\begin{enumerate}[label=(\arabic*)]
\item $A$出现,$B$,$C$不出现;
\item $A$,$B$都出现,$C$不出现;
\item 三个事件都出现;
\item 三个事件至少有一个出现;
\item 三个事件都不出现;
\item 不多于一个事件出现;
\item 不多于两个事件出现;
\item 三个事件至少有两个出现;
\item $A$,$B$至少有一个出现,$C$不出现;
\item$A$,$B$,$C$中恰好有两个出现.
\end{enumerate}
\end{multicols}

解答:

\begin{enumerate}[label=(\arabic*)]
\item $A$出现用$A$表示,$B$,$C$不出现用其对立事件表示,且同时发生,为积事件,所以表示为$A\overline{B}\,\overline{C}$;
\item $A$,mB都出现,$C$不出现同时发生,为积事件,所以表示为$AB\overline{C}$;
\item 三个事件都出现同时发生,为积事件,所以表示为$ABC$;
\item 三个事件至少有一个出现,为和事件,所以表示为$A\cup B\cup C$;
\item 三个事件都不出现同时发生,为积事件,所以表示为$\overline{A}\,\overline{B}\,\overline{C}$;
\item 不多于一个事件出现的对立事件是至少出现两个事件,所以表示为$\overline{(AB) \cup (AC) \cup (BC)}$;
\item 不多于两个事件出现的情况包含:三个事件都不出现、出现一个事件、出现两个事件,情况复杂;而该事件的对立事件仅有一种情况:三个时间都出现。因此,用简单事件的对立事件来表示。所以表示为$\overline{ABC}$;
\item 三个事件至少有两个出现是和事件,所以表示为$(AB)$;
\item $A$,$B$至少有一个出现是和事件,同时$C$不出现为积事件,所以表示为$(A \cup B) C$;
\item 恰好有两个出现有三种互斥的情况,所以表示为$(AB\bar{C}) \cup (A\bar{B}C) \cup (\bar{A}BC)$..
\end{enumerate}

点评

对于第6题,$(AB) \cup (AC) \cup (BC) \cup (ABC)$这个答案是冗余的。因为:

\begin{enumerate}
\item $ABC \subset AB$
\item $ABC \subset AC$
\item $ABC \subset BC$
\end{enumerate}

因此,将 $(ABC)$ 添加到并集不会改变任何内容 — — 它已经包含在内了。$(AB) \cup (AC) \cup (BC) \cup (ABC)$在逻辑上是多余的,但并非不正确。

对于第10题,$((AB) \cup (AC) \cup (BC)) \cap (ABC)^c$也是正确的,它和$(AB\bar{C}) \cup (A\bar{B}C) \cup (\bar{A}BC)$是等价的。

We start with:

$$
((AB) \cup (AC) \cup (BC)) \cap (ABC)^c
$$

This is the intersection of a union of 3 terms with the complement of $ABC$.

Let’s denote:

\begin{enumerate}
\item $X = AB$
\item $Y = AC$
\item $Z = BC$
\end{enumerate}

So we have:
$$
(X \cup Y \cup Z) \cap (ABC)^c
$$

Now apply the \textbf{distributive law of set intersection over union}:

$$
(X \cup Y \cup Z) \cap (ABC)^c = (X \cap (ABC)^c) \cup (Y \cap (ABC)^c) \cup (Z \cap (ABC)^c)
$$

That is:

$$
(AB \cap (ABC)^c) \cup (AC \cap (ABC)^c) \cup (BC \cap (ABC)^c)
$$

Let’s now simplify each of the three terms.

\textbf{Simplify each term}:

1. $AB \cap (ABC)^c$

Since $ABC \subset AB$, this becomes:

$$
AB \setminus ABC = AB \cap C^c
$$

2. $AC \cap (ABC)^c = AC \setminus ABC = AC \cap B^c$

3. $BC \cap (ABC)^c = BC \setminus ABC = BC \cap A^c$

Final Result:

$$
(AB \cap C^c) \cup (AC \cap B^c) \cup (BC \cap A^c)
$$

\separatorline

Here's the explaination of $\setminus$.

1. What does the symbol “$\setminus$” mean?

The symbol $\setminus$ denotes \textbf{set difference} (also called \textbf{relative complement}).

\textbf{Definition}:

For two events (sets) $X$ and $Y$,

$$
X \setminus Y = { \omega \in X : \omega \notin Y } = X \cap Y^c
$$

That is: the set of elements that are in $X$ \textbf{but not} in $Y$.


2. Apply this to $AB \setminus ABC$

Here:
\begin{itemize}
\item 
$AB = A \cap B$

\item 
$ABC = A \cap B \cap C$

\end{itemize}

So:

$$
AB \setminus ABC = AB \cap (ABC)^c
$$

This says: keep only those elements in $A \cap B$ \textbf{that are not} in $A \cap B \cap C$. In other words, all outcomes where $A$ and $B$ occur, but \textbf{not} $C$.

\hrulefill

3. Simplify $AB \cap (ABC)^c$

Since $ABC = A \cap B \cap C$, its complement is:

$$
(ABC)^c = (A \cap B \cap C)^c
$$

Now, intersecting with $AB$:

$$
AB \cap (A \cap B \cap C)^c
$$

Note:
\begin{itemize}
\item 
$AB$ is a subset of $A \cap B$, so we can factor:

\end{itemize}

$$
AB \cap (ABC)^c = AB \cap C^c
$$

This makes sense: from $AB$, we remove those outcomes where $C$ is also true — i.e., we keep only those with $A$ \textbf{and} $B$ \textbf{occurring}, but \textbf{not} $C$.

\hrulefill

Final Result:

$$
AB \setminus ABC = AB \cap C^c
$$

is valid because it literally means:

Take the set where both $A$ and $B$ occur, and \textbf{exclude} the outcomes where $C$ also occurs.

\hrulefill

Summary:

\begin{tabular}{l l}
Expression & Meaning \\
\hline
$X \setminus Y$ & Elements in $X$ but not in $Y$, or $X \cap Y^c$ \\
$AB \setminus ABC$ & Outcomes where $A$ and $B$ occur, but $C$ does not \\
Simplified as & $AB \cap C^c$ \\
\end{tabular}

\section{全概率公式}

全概率公式的核心思想是:\textbf{将一个复杂事件分解为多个互斥且完备的小事件},然后通过这些小事件的概率来计算复杂事件的概率。

公式如下:

\[
P(A) = \sum_{i=1}^n P(A \mid B_i) \cdot P(B_i)
\]

其中:
\begin{itemize}
\item \( B_1, B_2, \dots, B_n \) 是互斥且完备的事件(即它们覆盖所有可能性,且两两不相交)。
\item \( P(A \mid B_i) \) 是在事件 \( B_i \) 发生的条件下,事件 \( A \) 发生的概率。
\end{itemize}

\subsection*{例子:掷骰子}

假设我们有一个公平的六面骰子,编号为 1 到 6。我们定义:

\begin{itemize}
\item 事件 \( A \):掷出的点数是偶数。
\item 事件 \( B_1 \):掷出的点数是 1、2 或 3。
\item 事件 \( B_2 \):掷出的点数是 4、5 或 6。
\end{itemize}

\textbf{问题}:求事件 \( A \) 的概率 \( P(A) \)。

\separatorline

\textbf{步骤 1:分解事件}

首先,我们将事件 \( A \) 分解为两个互斥且完备的事件:

\begin{enumerate}[label=\arabic*.]
\item 事件 \( A \) 且 \( B_1 \) 发生:掷出 2。
\item 事件 \( A \) 且 \( B_2 \) 发生:掷出 4 或 6。
\end{enumerate}

\textbf{步骤 2:计算条件概率}

\begin{itemize}
\item \( P(A \mid B_1) \):在 \( B_1 \) 发生的条件下,\( A \) 发生的概率。
	\begin{itemize}
		\item \( B_1 \) 包含的点数是 1、2、3,其中只有 2 是偶数。
		\item 所以 \( P(A \mid B_1) = \frac{1}{3} \)。
	\end{itemize}
\item \( P(A \mid B_2) \):在 \( B_2 \) 发生的条件下,\( A \) 发生的概率。
	\begin{itemize}
		\item \( B_2 \) 包含的点数是 4、5、6,其中 4 和 6 是偶数。
		\item 所以 \( P(A \mid B_2) = \frac{2}{3} \)。
	\end{itemize}
\end{itemize}

\textbf{步骤 3:计算 \( B_1 \) 和 \( B_2 \) 的概率}

\begin{itemize}
\item \( P(B_1) = \frac{3}{6} = \frac{1}{2} \)。
\item \( P(B_2) = \frac{3}{6} = \frac{1}{2} \)。
\end{itemize}

\textbf{步骤 4:应用全概率公式}

根据全概率公式:

\[
P(A) = P(A \mid B_1) \cdot P(B_1) + P(A \mid B_2) \cdot P(B_2)
\]

代入数值:

\[
P(A) = \left(\frac{1}{3}\right) \cdot \left(\frac{1}{2}\right) + \left(\frac{2}{3}\right) \cdot \left(\frac{1}{2}\right) = \frac{1}{6} + \frac{2}{6} = \frac{3}{6} = \frac{1}{2}
\]

\subsection*{例子:抽奖问题}

假设有一个抽奖箱,里面有 3 个盒子:
\begin{itemize}
\item 
盒子 1:2 个红球,3 个蓝球。

\item 
盒子 2:4 个红球,1 个蓝球。

\item 
盒子 3:3 个红球,2 个蓝球。
\end{itemize}

规则:
\begin{itemize}
\item 
先随机选择一个盒子(选择每个盒子的概率相等,即 \( \frac{1}{3} \))。

\item 
然后从选中的盒子中随机抽一个球。
\end{itemize}

问题:求抽到红球的概率 \( P(\text{红球}) \)。

\separatorline

\textbf{步骤 1:定义事件}

\begin{itemize}
\item 
设 \( B_1 \) 为选择盒子 1,\( B_2 \) 为选择盒子 2,\( B_3 \) 为选择盒子 3。

\item 
设 \( A \) 为抽到红球。

\end{itemize}


\textbf{步骤 2:计算条件概率}

\begin{itemize}
\item 
在盒子 1 中,红球的概率是:
\[
P(A \mid B_1) = \frac{2}{2+3} = \frac{2}{5}
\]

\item 
在盒子 2 中,红球的概率是:
\[
P(A \mid B_2) = \frac{4}{4+1} = \frac{4}{5}
\]

\item 
在盒子 3 中,红球的概率是:
\[
P(A \mid B_3) = \frac{3}{3+2} = \frac{3}{5}
\]
\end{itemize}

\textbf{步骤 3:计算选择盒子的概率}

每个盒子被选中的概率相等:
\[
P(B_1) = P(B_2) = P(B_3) = \frac{1}{3}
\]

\textbf{步骤 4:应用全概率公式}

根据全概率公式:

\[
P(A) = P(A \mid B_1) \cdot P(B_1) + P(A \mid B_2) \cdot P(B_2) + P(A \mid B_3) \cdot P(B_3)
\]

代入数值:

\[
P(A) = \left(\frac{2}{5}\right) \cdot \left(\frac{1}{3}\right) + \left(\frac{4}{5}\right) \cdot \left(\frac{1}{3}\right) + \left(\frac{3}{5}\right) \cdot \left(\frac{1}{3}\right)
\]

计算:

\[
P(A) = \frac{2}{15} + \frac{4}{15} + \frac{3}{15} = \frac{9}{15} = \frac{3}{5}
\]

\subsection*{例子:疾病检测}

假设某疾病的发病率为 1\%(即 \( P(D) = 0.01 \)),检测方法的准确率为:

\begin{itemize}
\item 如果患病,检测结果为阳性的概率是 99\%(即 \( P(T^+ \mid D) = 0.99 \))。
\item 如果未患病,检测结果为阳性的概率是 5\%(即 \( P(T^+ \mid \neg D) = 0.05 \))。
\end{itemize}

\textbf{问题}:求检测结果为阳性的概率 \( P(T^+) \)。

\separatorline

\textbf{步骤 1:分解事件}

事件 \( T^+ \) 可以分解为:

\begin{enumerate}[label=\arabic*.]
\item 患病且检测为阳性:\( D \cap T^+ \)。
\item 未患病且检测为阳性:\( \neg D \cap T^+ \)。
\end{enumerate}

\textbf{步骤 2:应用全概率公式}

\[
P(T^+) = P(T^+ \mid D) \cdot P(D) + P(T^+ \mid \neg D) \cdot P(\neg D)
\]

代入数值:

\[
P(T^+) = (0.99 \cdot 0.01) + (0.05 \cdot 0.99) = 0.0099 + 0.0495 = 0.0594
\]

% pip install Pygments latexminted
% 如果出现错误,记得删除_minted文件夹

\subsubsection*{代码模拟}

使用蒙特卡洛模拟(即随机模拟)的方式来估算这个概率。

\begin{minted}{python}
import random

def simulate_positive_test(num_simulations):
    positive_count = 0

    for _ in range(num_simulations):
        # 是否患病
        if random.random() < 0.01:
            # 患病,检查是否阳性
            if random.random() < 0.99:
                positive_count += 1
        else:
            # 未患病,检查是否阳性
            if random.random() < 0.05:
                positive_count += 1

    return positive_count / num_simulations

# 设置模拟次数
num_simulations = int(input("请输入模拟次数:"))
p = simulate_positive_test(num_simulations)

print(f"在{num_simulations}次模拟中, 检测结果为阳性的概率约为: {p:.4f}")
\end{minted}

\textbf{总结}

全概率公式的核心思想是:\textbf{将复杂事件分解为多个互斥且完备的小事件},然后通过条件概率和边缘概率计算复杂事件的概率。通过简单的例子(如掷骰子)和复杂的例子(如疾病检测),我们可以更好地理解和应用这一公式。

\end{document}